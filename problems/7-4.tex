\begin{problem}[习题7.4]
对于方程
\[
\ddot{u} + \omega_0^2 u + \varepsilon \dot{u}^3 = 0
\]
试求$u$的一阶的一致有效展开式.
\end{problem}

\begin{solution}
\textbf{解:} 当$\varepsilon = 0$时, 原方程的退化方程的通解为
\begin{equation}\label{s20}
u = a\cos(\omega_0 t + \theta)
\end{equation}
其中$a$和$\theta$是常数, 分别是振幅和初始位相.  由\textbf{平均法}, 当$0<\varepsilon\ll 1$时, 可认为解的形式仍为式(\ref{s20}). 但$a$和$\theta$不是常数, 而是$t$的缓变函数. 原式中
\[
\ddot{u} + \omega_0^2 u = - \varepsilon \dot{u}^3 = \varepsilon f(u, du/dt)
\]
故$f(u,\dot{u})$的形式为$-\dot{u}^3$, 故有
\[
f(a\cos\varphi, -a\omega_0\sin\varphi) = -(-a\omega_0\sin\varphi)^3
\]
因此由平均法有
\begin{eqnarray}
\frac{da}{dt}
&=& \frac{\varepsilon}{~~\omega_0}\frac{1}{2\pi}\int_0^{2\pi}
\sin\varphi (-a\omega_0\sin\varphi)^3 d\varphi = -\frac{3}{8}a^3\omega_0^2\varepsilon \nonumber\\
\noalign{\vskip5pt}
\frac{d\theta}{dt}
&=& \frac{\varepsilon}{a\omega_0}\frac{1}{2\pi}\int_0^{2\pi}
\cos\varphi (-a\omega_0\sin\varphi)^3 d\varphi = 0 \nonumber
\end{eqnarray}
因此可解出$a$和$\theta$:
\[
a = \Big(\frac{3}{4}\varepsilon\omega_0^2 t + \frac{1}{a_0^2}\Big)^{-1/2}, ~~ \theta = \theta_0
\]
将$a$和$\theta$代入式(\ref{s20}), 可得方程的一致有效展开式为
\[
u = \Big(\frac{3}{4}\varepsilon\omega_0^2 t + \frac{1}{a_0^2}\Big)^{-1/2}\cos(\omega_0 t + \theta_0)
\]
其中$a_0$和$\theta_0$是常数.

\end{solution} 
