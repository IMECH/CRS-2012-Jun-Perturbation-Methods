\begin{problem}[习题7.26]
试求边值问题
\[
\varepsilon y''+y'=2x, ~~y(0)=\alpha, ~~y(1)=\beta
\]
的一致有效展开式.
\end{problem}

\begin{solution}
%由习题(7.19)及$y'$前的系数1大于0可知, 该问题的边界层在$x=0$处. 先求外解, 由原方程得
%\[
%{y'}_{0}^{(o)}=2x,y_{0}^{(o)}(1)=\beta
%\]
%由上式及边界条件可求得外解为
%\[
%y_{0}^{(o)}=x^{2}+\beta-1
%\]
%设内变量为$\xi=x/\varepsilon^{\lambda}$, 代入原方程式
%\[
%\varepsilon^{1-2\lambda}\frac{d^{2}y^{(i)}}{d\xi^{2}}+\varepsilon^{-\lambda}\frac{dy^{(i)}}{d\xi}=2\xi\varepsilon^{\lambda}
%\]
%即
%\[
%\frac{d^{2}y^{(i)}}{d\xi^{2}}+\varepsilon^{\lambda-1}\frac{dy^{(i)}}{d\xi}=2\xi\varepsilon^{3\lambda-1}
%\]
%由于外解忽略了第一项, 所以在边界层中应该包含第一项, 因此$\lambda=1$, 此时上式化为
%\[
%\frac{d^{2}y^{(i)}}{d\xi^{2}}+\frac{dy^{(i)}}{d\xi}=0\Rightarrow y^{(i)}=C_{0}e^{-\xi}+C_{1}
%\]
%由边界条件$y^{(i)}(0)=\alpha$可得$C_{0}+C_{1}=\alpha$, 因此有$y^{(i)}=C_{0}e^{-\xi}+(\alpha-C_{0})$.
%
%\vspace{1em}
%
%\noindent 对内外解应用Prandtl匹配原理($\lim_{x\rightarrow0}y^{(o)}=\lim_{\xi\rightarrow\infty}y^{(i)}$)得
%\[
%\beta-1=\alpha-C_{0}\Rightarrow C_{0}=\alpha-\beta+1
%\]
%因此外解为
%\[
%y^{(i)}=(\alpha-\beta)e^{\xi}+(\beta-1)
%\]
%综上所述, 最终解为
%\[
%y=y^{(0)}+y^{(i)}-[y^{(o)}]^{(i)}=x^{2}+\beta-1+(\alpha-\beta)e^{\xi}
%\]

由习题(7.19)及$y'$前的系数1大于0可知, 该问题的边界层在$x=0$处. 先求外解, 设外解形式为$y^{(o)}=y_{0}^{(o)}+\delta_{1}(\varepsilon)y_{1}^{(o)}+\cdots$,
且$\delta_{1}(\varepsilon\rightarrow0)\rightarrow0$. 对于本题取$\delta_{1}(\varepsilon)=\varepsilon$.
由原方程得
\[
{y'}_{0}^{(o)}=2x,~~y_{0}^{(o)}(1)=\beta
\]
\[
{y''}_{0}^{(o)}+{y'}_{1}^{(o)}=0
\]
由上式及边界条件可求得外解为
\[
y_{0}^{(o)}=x^{2}+\beta-1,y_{1}^{(o)}=-2x+C_{1}^{(o)}
~\rightarrow~
y^{(0)}=x^{2}+\beta-1+\varepsilon\big(-2x+C_{1}^{(o)}\big)
\]
设内变量为$\xi=x/\varepsilon^{\lambda}$, 代入原方程式
\[
\varepsilon^{1-2\lambda}\frac{d^{2}y^{(i)}}{d\xi^{2}}+\varepsilon^{-\lambda}\frac{dy^{(i)}}{d\xi}=2\xi\varepsilon^{\lambda}
~\Rightarrow~
\frac{d^{2}y^{(i)}}{d\xi^{2}}+\varepsilon^{\lambda-1}\frac{dy^{(i)}}{d\xi}=2\xi\varepsilon^{3\lambda-1}
\]
由于外解忽略了第一项, 所以在边界层中应该包含第一项, 因此$\lambda=1$, 此时上式化为
\begin{equation}
\frac{d^{2}y^{(i)}}{d\xi^{2}}+\frac{dy^{(i)}}{d\xi}=2\xi\varepsilon^{2}\label{eq:72601}
\end{equation}
显然内展开的形式应为
\[
y^{(i)}=y_{0}^{(i)}(\xi)+\varepsilon^{2}y_{1}^{(i)}(\xi)+\cdots
\]
代入式(\ref{eq:72601}), 令$\text{\ensuremath{\varepsilon}}$的同次幂的系数相等有
\[
\frac{d^{2}y_{0}^{(i)}}{d\xi^{2}}+\frac{dy_{0}^{(i)}}{d\xi}=0, ~~ \frac{d^{2}y_{1}^{(i)}}{d\xi^{2}}+\frac{dy_{1}^{(i)}}{d\xi}=2\xi
\]
由$x=0$处的边界条件$y_{0}^{(i)}(0)=\alpha$, $y_{1}^{(i)}(0)=0$可得零阶和一阶内解为
\[
y_{0}^{(i)}=C_{0}^{(i)}e^{-\xi}+(\alpha-C_{0}^{(i)}), ~~ y_{1}^{(i)}=C_{1}^{(i)}e^{-\xi}-C_{1}^{(i)}+\xi^{2}-2\xi
\]
现在按 Van Dyke 匹配原理进行匹配($m=n=3$).
\begin{itemize}
\item 两项外展开式: \ \  $x^{2}+\beta-1+\varepsilon(-2x+C_{1}^{(o)})$,
\\改写为内变量: \ \  $\varepsilon^{2}\xi^{2} + \beta-1 + \varepsilon (-2\varepsilon\xi+C_{1}^{(o)})$ ,\\
取三项内展开: \ \  $(\beta-1) + \varepsilon C_{1}^{(o)} + \varepsilon^{2}(\xi^{2}-2\xi)$, \\
改写为外变量: \ \  $x^{2}+\beta-1-2\varepsilon x+\varepsilon C_{1}^{(o)}$.
\item 两项内展开式: \ \  $C_{0}^{(i)}e^{-\xi} + (\alpha-C_{0}^{(i)}) + \varepsilon^{2}\big(C_{1}^{(i)}e^{-\xi} - C_{1}^{(i)} + \xi^{2} - 2\xi\big)$,\\
改写为外变量: \ \  $C_{0}^{(i)}e^{-x/\varepsilon} + (\alpha-C_{0}^{(i)}) + \varepsilon^{2}C_{1}^{(i)}e^{-x/\varepsilon} - \varepsilon^{2}C_{1}^{(i)} + x^{2} - 2x\varepsilon$,\\
取三项外展开: \ \  $(\alpha - C_{0}^{(i)}) + x^{2} - 2x\varepsilon - \varepsilon^{2}C_{1}^{(i)}$.
\end{itemize}
应用 Van Dyke 匹配原理可得
\[
\beta-1=\alpha-C_{0}^{(i)},C_{1}^{(o)}=C_{1}^{(i)}=0
\]
因此外内解为
\[
y^{(o)}=x^{2}+\beta-1-2x\varepsilon
\]
\[
y^{(i)}=(\alpha-\beta+1)e^{-\xi}+(\beta-1)+\varepsilon^{2}\big(\xi^{2}-2\xi\big)
\]
综上所述, 最终解为
\[
y=y^{(0)}+y^{(i)}-[y^{(i)}]^{(o)}=x^{2}-2x\varepsilon+(\alpha-\beta+1)e^{-x/\varepsilon}+(\beta-1)
\]

%对内外解应用Prandtl
%
%匹配原理($\lim_{x\rightarrow0}y^{(o)}=\lim_{\xi\rightarrow\infty}y^{(i)}$)得
%\[
%\beta-1=\alpha-C_{0}\Rightarrow C_{0}=\alpha-\beta+1
%\]
%因此外解为
%\[
%y^{(i)}=(\alpha-\beta)e^{\xi}+(\beta-1)
%\]
%综上所述, 最终解为
%\[
%y=y^{(0)}+y^{(i)}-[y^{(o)}]^{(i)}=x^{2}+\beta-1+(\alpha-\beta)e^{\xi}
%\]
\end{solution} 
