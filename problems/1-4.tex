\begin{problem}[习题1.4]
试用量阶符号的定义, 证明量阶运算的性质:
\begin{enumerate}
\item $O(\psi) + O(\psi) = O(\psi)$
\item $O(\psi) \cdot o(\varphi) = o(\psi\varphi)$
\end{enumerate}
\end{problem}

\begin{solution}
\begin{enumerate}
\item 根据量阶符号的定义, $\exists A,B$常数使得
\[
|O(\psi)+O(\psi)|\leq|O(\psi)|+|O(\psi)|\leq A|\psi|+B|\psi|=(A+B)|\psi|
\]
其中$A+B$为常数. 因此有$O(\psi)+O(\psi)=O(\psi)$.
\item 根据量阶符号的定义, $\exists A$常数及任意$\varepsilon>0$有
\[
|O(\psi)\cdot o(\varphi)|=|O(\psi)|\cdot|o(\varphi)|\leq A|\psi|\cdot\varepsilon|\varphi|=\varepsilon A|\psi\varphi|
\]
其中$\varepsilon A$为任意小的正数. 因此有$O(\psi)\cdot o(\varphi)=o(\psi\varphi)$.\end{enumerate}
\end{solution} 
